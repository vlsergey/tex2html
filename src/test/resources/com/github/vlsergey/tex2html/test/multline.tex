\documentclass{article}
\begin{document}
Операция <<добавление ключа раунда>> состоит в том, что матрица текущего состояния складывается по модулю 2 с матрицей ключа текущего раунда. Обе матрицы должны иметь одинаковые размеры. Матрица ключа раунда вычисляется с помощью процедуры \emph{расширения ключа}, описанной ниже. Операция <<добавление ключа раунда>> обозначается $\mathsf{AddRoundKey(State, RoundKey)}$.

\begin{multline*}
    \left[ \begin{array}{cccc}
        a_{0,0} & a_{0,1} & a_{0,2} & a_{0,3} \\
        a_{1,0} & a_{1,1} & a_{1,2} & a_{1,3} \\
        a_{2,0} & a_{2,1} & a_{2,2} & a_{2,3} \\
        a_{3,0} & a_{3,1} & a_{3,2} & a_{3,3}
    \end{array} \right]
    \oplus
    \left[ \begin{array}{cccc}
        k_{0,0} & k_{0,1} & k_{0,2} & k_{0,3} \\
        k_{1,0} & k_{1,1} & k_{1,2} & k_{1,3} \\
        k_{2,0} & k_{2,1} & k_{2,2} & k_{2,3} \\
        k_{3,0} & k_{3,1} & k_{3,2} & k_{3,3}
    \end{array} \right] =
    \\
    = \left[ \begin{array}{cccc}
        b_{0,0} & b_{0,1} & b_{0,2} & b_{0,3} \\
        b_{1,0} & b_{1,1} & b_{1,2} & b_{1,3} \\
        b_{2,0} & b_{2,1} & b_{2,2} & b_{2,3} \\
        b_{3,0} & b_{3,1} & b_{3,2} & b_{3,3}
    \end{array} \right].
\end{multline*}
\end{document}
